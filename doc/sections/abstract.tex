%%%%%%%%%%%%%%%%%%%%%%% Uncomment for the Greek Version %%%%%%%%%%%%%%%%%%%%%%%
% \begin{center}
%     \Large{\textbf{Ο Τίτλος Μου}} \\~\\
%     \Large{\textbf{Από}} \\~\\
%     \Large{\textbf{Όνομα Επώνυμο}} \\~\\
    
%     \large{Υποβλήθηκε στο ΔΠΜΣ «Τεχνητή Νοημοσύνη» την \\ ΧΧ Μηνός 20ΧΧ ως \\ υποχρέωση για την λήψη Μεταπτυχιακού Διπλώματος Σπουδών  \\~\\}
% \end{center}

% \renewenvironment{abstract}
%  {\par\noindent\textbf{Περίληψη}\ \ignorespaces}
%  {\par\medskip}

% \begin{abstract}
%     \par Η περίληψη πρέπει να περιλαμβάνει παρουσίαση του προβλήματος που η διπλωματική αντιμετώπισε, τις μεθόδους που ανέπτυξε, τα σημεία συνεισφοράς της, και πως αυτά αναδεικνύονται/αποδεικνύονται από πειραματικά ή/και θεωρητικά αποτελέσματα. \\~\\

%     \begin{flushleft}
%       Επιβλέπων/Επιβλέπουσα:  \\
%       Ακαδημαϊκή Θέση: \\
%     \end{flushleft}
% \end{abstract}

\begin{center}
    \Large{\textbf{My Title}} \\~\\
    \Large{\textbf{By}} \\~\\
    \Large{\textbf{Name Surname}} \\~\\
    
    \large{Submitted to the II-MSc “Artificial Intelligence” on \\ Month XX, 20XX, \\ in partial fulfillment of the \\ requirements for the MSc degree \\~\\}
\end{center}

\renewenvironment{abstract}
 {\par\noindent\textbf{\abstractname}\ \ignorespaces}
 {\par\medskip}

\begin{abstract}
    \\ A small abstract describing the problem addressed, the methods developed, the contributions made and how contributions have are justified through experimental and/or theoretical means. 
    Not more than 20 lines of text. \\~\\
    
    \begin{flushleft}
        Thesis Supervisor:  \\
        Title: \\
    \end{flushleft}
\end{abstract}