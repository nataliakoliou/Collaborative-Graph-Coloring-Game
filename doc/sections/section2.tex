\subsection{Formatting}

\begin{flushleft}
    Μεταξύ δύο διαδοχικών τίτλων, ανεξαρτήτως επιπέδου θα πρέπει να υπάρχει κάποιο εισαγωγικό κείμενο 2-3 σειρών (που συνήθως προλογίζει όσα ακολουθούν).  \\~\\
\end{flushleft}

\subsubsection{Settings - Paging}

\begin{flushleft}
    Το τελικό κείμενο θα τυπωθεί σε χαρτί μεγέθους Α4, με εκτύπωση διπλής όψης (εμπρός + πίσω). Τα περιθώρια δεξιά, αριστερά, πάνω και κάτω από το κείμενο είναι 25mm. Επίσης υπάρχει πρόβλεψη για τη βιβλιοδεσία πλάτους 10mm. Το κείμενο θα πρέπει να έχει πλήρη στοίχιση με χρήση συλλαβισμού προκειμένου να αποφεύγονται τα άσχημα μεγάλα κενά στις σειρές (είναι όλα ρυθμισμένα στο παρόν αρχείο). \\~\\

    Πληροφοριακά, το κυρίως κείμενο είναι σε γραμματοσειρά Georgia με μέγεθος 12 pts και διάστιχο 1½ γραμμής. Για έμφαση του κειμένου θα πρέπει να χρησιμοποιείται μόνο η πλαγιαστή γραφή και ΟΧΙ η έντονη ή η υπογραμμισμένη. \\~\\
    
    Το παρόν αρχείο είναι σελιδοποιημένο για εκτύπωση διπλής όψης και επιπλέον περιέχει αρίθμηση. Προφανώς, μέχρι την τελική εκτύπωση, μπορείτε να το τυπώνεται και σε εκτύπωση μονής όψης. \\~\\
\end{flushleft}


\subsubsection{Use of Styles}

\begin{flushleft}
    Για ομοιόμορφη μορφοποίηση θα πρέπει να χρησιμοποιήσετε τα styles που περιέχει το παρόν αρχείο. Τα σημαντικότερα από αυτά είναι: \\~\\

    Το στυλ Normal (Βασικό) για το βασικό κείμενο \\~\\
    
    Το στυλ Heading 1 (Επικεφαλίδα 1) για την επικεφαλίδα κεφαλαίου, το στυλ Heading 2 (Επικεφαλίδα 2) για την επικεφαλίδα ενότητας, κ.ο.κ. Μπορείτε να χρησιμοποιήσετε μέχρι και το Heading 5. Τα Heading 1 ως 3 έχουν αυτόματη αρίθμηση. Τα 4 και 5 είναι χωρίς αρίθμηση. \\~\\
    
    Το στυλ Caption (Λεζάντα) που μπαίνει αυτόματα όταν φτιάχνετε λεζάντες. \\~\\
    
    Τα παραπάνω στυλ είναι ήδη ρυθμισμένα στο παρόν αρχείο, οπότε απλά τα χρησιμοποιείτε.  \\~\\
    
    Μετά την τελευταία παράγραφο ενότητας, όπως η παρούσα, δεν αφήνουμε γενικά κενές γραμμές καθώς οι παράγραφοι με τους τίτλους είναι ρυθμισμένες έτσι ώστε να δεσμεύουν τον απαιτούμενο χώρο.  \\~\\
\end{flushleft}

\subsubsection{Figures and Tables}

\paragraph{Figures}

\begin{flushleft}
    Οι εικόνες μπαίνουν in-line και από κάτω έχουν λεζάντα. Για όλες τις εικόνες θα πρέπει να υπάρχει τουλάχιστον μία αναφορά μέσα στο κείμενο. Οι εικόνες θα πρέπει να βρίσκονται κοντά στο κείμενο στο οποίο γίνεται αναφορά σε αυτές και συνήθως μετά από αυτό το κείμενο. Ακολουθεί ένα παράδειγμα τέτοιας αναφορά. \\~\\
\end{flushleft}

\paragraph{Tables}

\begin{flushleft}
    Η λεζάντα στον πίνακα μπαίνει στο πάνω μέρος. Μετά τον πίνακα αφήνουμε μία κενή σειρά, όπως στο παράδειγμα.  \\~\\
    
    \begin{table}
        \centering
        \renewcommand\tablename{Πίνακας}
        \caption{\label{tab:table-name}Δοκιμαστικός πίνακας. }
        \begin{tabular}{|c|c|c|c|}
         \hline
          & Ελλάδα & Αγγλία & Γαλλία\\
         \hline\hline
         Πληθυσμός & 10 εκ.  & 55 εκ. & 60 εκ. \\ 
         \hline
         Έκταση & 132000 τ.χ. & 800000 τ.χ. & 800000 τ.χ. \\
         \hline
        \end{tabular}
    \end{table}
    
    Όπως και με τις εικόνες, θα πρέπει να γίνεται αναφορά κάπου μέσα στο κείμενο για τον εκάστοτε πίνακα. Συνίσταται και εδώ η χρήση παραπομπών (cross-reference). \\~\\
\end{flushleft}

\subsubsection{Titles}

\begin{flushleft}
    Οι τίτλοι θα πρέπει να είναι μικροί σε μέγεθος και περιεκτικοί ως προς το περιεχόμενο. Δεν θα πρέπει να ξεπερνούν γενικά τη μία γραμμή. Για τη μορφοποίηση υπάρχουν σχετικά στυλ, όπως αναφέρθηκε. Τα κεφάλαια αρχίζουν σε δεξιά σελίδα, όταν η εκτύπωση είναι διπλής όψης. 
\end{flushleft}

\subsubsection{References}

\begin{flushleft}
    Στο τέλος της διπλωματικής θα πρέπει να υπάρχει αριθμημένη βιβλιογραφία. Μέσα στο κείμενο θα πρέπει να βάζετε αναφορά στον αριθμό της πηγής που βρίσκεται στην βιβλιογραφία όπου αυτό είναι απαραίτητο. Η αναφορά αυτή θα γίνεται βάζοντας τον αριθμό της πηγής μέσα σε αγκύλες, π.χ. [1], [2, 3], κτλ... 
\end{flushleft}

\subsubsection{Final Draft}

\begin{flushleft}
    Στο τέλος θα παραδοθούν 1 πρωτότυπο και δύο αντίγραφα της διπλωματικής. Το πρωτότυπο θα έχει επισυναπτόμενο ένα CD που θα περιέχει τον κώδικα, το κείμενο της διπλωματικής και την παρουσίαση της διπλωματικής. 
\end{flushleft}