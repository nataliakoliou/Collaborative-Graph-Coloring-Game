\subsection{Introduction to Game Theory and Analysis}

\begin{flushleft}

    Game theory is the mathematical study of strategic decision-making in situations where independent, self-interested agents interact with one another \cite{Shoham_Leyton-Brown_2008}. It provides a structured way to model strategic behaviors, with the goal of understanding how choices affect the agents' outcomes. The key assumption in game theory is that agents are rational, meaning they make decisions that maximize their individual payoffs. By making this assumption, we can identify equilibrium points—strategies or behaviors that no agent has an incentive to deviate from. In Game Theory, the focus can be either on direct outcomes, involving strategies, or indirect outcomes, involving policies.\\~\\

    The study of strategic decision-making is divided into two main fields: Classical Game Theory (CGT) and Evolutionary Game Theory (EGT). Classical Game Theory focuses on games with actual players and their strategies. A key solution concept in CGT is the \emph{Nash equilibrium}, which identifies the strategy profiles in which no player can improve their outcome by unilaterally changing their strategy, assuming the strategies of others remain unchanged \cite{doi:10.1073/pnas.36.1.48}. On the other hand, Evolutionary Game Theory examines indirect outcomes that occur when players adopt policies, also known as styles of play, rather than specific strategies. Here, equilibrium is based on the evolution of behaviors over time, often described using concepts like \emph{Markov-Conley chains} (MCCs). In MCCs, the long-term dynamics of agent interactions are captured in a strategy space, where each point represents a possible strategy. The graph formed by these strategies consists of strongly connected components, where each node represents a strategy, and edges represent the transitions between strategies. Once players enter these components, they tend to remain, indicating equilibrium \cite{omidshafiei2019alpharank}.\\~\\

    In static games, where payoff matrices are known, finding \emph{Nash equilibria} is relatively straightforward. For example, consider the payoff matrix for the Rock-Paper-Scissors game in Table~\ref{tab:rps_payoff}. The mixed-strategy \emph{Nash equilibrium} occurs when both players randomize their choices uniformly across Rock, Paper, and Scissors. However, in dynamic settings, where decision making is sequential, one must account for the dynamics of agents' interactions over time. In these settings, we need to analyze agents' behavior in terms of their payoffs, identifying joint strategies that result into agents' stable behaviors. To illustrate how complexity increases as a game unfolds over time, consider a repeated version of the Rock-Paper-Scissors game played over two stages. The payoff matrix for this 2-stage game, as shown in Table~\ref{tab:rps_2stage_payoff}, is significantly larger than in the static case. 
    
    \begin{table}[t]
        \centering
        \caption{Payoff matrix for the Rock-Paper-Scissors game.}
        \vspace{0.6em}
        \label{tab:rps_payoff}
        \begin{tabular}{c|c c c}
            & Rock & Paper & Scissors \\ \hline
            Rock     & 0,0    & -1,1   & 1,-1 \\
            Paper    & 1,-1   & 0,0    & -1,1 \\
            Scissors & -1,1   & 1,-1   & 0,0 \\
        \end{tabular}
    \end{table}
    
    The rows and columns represent the possible strategies that players can adopt across both stages, while the values in each cell reflect the sum of the payoffs from each individual round. In this case, the mixed-strategy Nash equilibrium involves both players randomizing equally among the strategies 
    \(R{-}R\), \(P{-}P\), and \(S{-}S\), each with a probability of \(\frac{1}{3}\), while assigning zero probability to all other strategies.\\~\\

    Although CGT provides a robust foundation for understanding static interactions, its solution concepts cannot easily reveal equilibria across sequences of actions; the number of possible policies is extremeley difficult to define, especially in games with many players that unfold over a large number of rounds. Evolutionary approaches have shown great potential towards this aim.\\~\\

    \begin{table}[t] 
        \centering 
        \caption{Payoff matrix for a 2-stage repeated Rock-Paper-Scissors game.}
        \vspace{0.6em}
        \label{tab:rps_2stage_payoff} 
        \begin{tabular}{c|c c c c c c c c c} 
                & R-R  & R-P  & R-S  & P-R  & P-P  & P-S  & S-R  & S-P  & S-S \\ \hline
            R-R & 0,0  & -1,1 & 1,-1 & -1,1 & -2,2 & 0,0  & 0,0  & 0,0  & -2,2 \\
            R-P & 1,-1 & 0,0  & -1,1 & 0,0  & -1,1 & -2,2 & 2,-2 & 1,-1 & 0,0 \\
            R-S & -1,1 & 1,-1 & 0,0  & -2,2 & 0,0  & -1,1 & 0,0  & 2,-2 & 1,-1 \\
            P-R & 1,-1 & 0,0  & 2,-2 & 0,0  & -1,1 & 1,-1 & -1,1 & 2,-2 & 0,0 \\
            P-P & 2,-2 & 1,-1 & 0,0  & 1,-1 & 0,0  & -1,1 & 0,0  & 1,-1 & -2,2 \\
            P-S & 0,0  & 2,-2 & 1,-1 & -1,1 & 1,-1 & 0,0  & -2,2 & 0,0  & -1,1 \\
            S-R & -1,1 & -2,2 & 0,0  & 1,-1 & 0,0  & 2,-2 & 0,0  & 1,-1 & 1,-1 \\
            S-P & 0,0  & -1,1 & -2,2 & 2,-2 & 1,-1 & 0,0  & 1,-1 & 0,0  & -1,1 \\
            S-S & -2,2 & 0,0  & -1,1 & 0,0  & 2,-2 & -1,1 & -1,1 & 1,-1 & 0,0 \\
    \end{tabular} 
\end{table}
    
\end{flushleft}

\begin{flushleft}
    ....
\end{flushleft}

\subsubsection{Dynamic Games}

\begin{flushleft}
    ....\\~\\

    ....\\~\\
    
    ....\\~\\
\end{flushleft}

\subsubsection{Empirical Games}