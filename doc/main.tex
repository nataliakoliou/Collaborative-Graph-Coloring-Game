\documentclass[12pt]{article}

\usepackage{geometry}
\geometry{
 a4paper,
 left=25mm,
 right=25mm,
 }

\usepackage[LGR, T1]{fontenc}
\usepackage{fontspec}
\setmainfont{Georgia}

\usepackage{graphicx,float}

\usepackage[toc,page]{appendix}

\usepackage{hyperref}

\usepackage{biblatex}
\addbibresource{example.bib}

%%%%%%%%%%%%%%%%%%%%%%% Uncomment for the Greek Version %%%%%%%%%%%%%%%%%%%%%%%
% \title{Τίτλος Διπλωματικής}
% \author{Από  \\~\\ \Large{Όνομα Επώνυμο} \\~\\}
% \date{Υποβάλλεται \\[12pt]
%       για την εκπλήρωση των προϋποθέσεων λήψης  \\[12pt]
%       Μεταπτυχιακού Διπλώματος  \\[12pt]
%       στην «Τεχνητή Νοημοσύνη» \\[12pt]
%       στο \\[12pt]
%       ΠΑΝΕΠΙΣΤΗΜΙΟ ΠΕΙΡΑΙΩΣ \\[15pt]
%       Μήνας 20ΧΧ \\[30pt]
%       \footnotesize{Πανεπιστήμιο Πειραιώς, ΕΚΕΦΕ «ΔΗΜΟΚΡΙΤΟΣ». Κάτοχος όλων των δικαιωμάτων} }

\title{MSc Thesis Title}
\author{by  \\~\\ \Large{Name Surname} \\~\\}
\date{Submitted \\[14pt]
      in partial fulfilment of the requirements for the degree of \\[14pt]
      Master of Artificial Intelligence \\[14pt]
      at the \\[14pt]
      UNIVERSITY OF PIRAEUS \\[20pt]
      Month 20ΧΧ \\[40pt]
      \footnotesize{University of Piraeus, NCSR “Demokritos”.  All rights              reserved.} }

\begin{document}

\begin{figure}
  \makebox[\textwidth][c]{\includegraphics[scale=0.5]{images/Header.png}}
\end{figure}

\maketitle
\thispagestyle{empty}
\newpage

%%%%%%%%%%%%%%%%%%%%%%% Uncomment for the Greek Version %%%%%%%%%%%%%%%%%%%%%%%
% Συγγραφέας . . . . . . . . . . . . . . . . . . . . . . . . . . . . . . . . . . . . . . . . . . . . .\\
% \begin{flushright}
%     ΔΠΜΣ «Τεχνητή Νοημοσύνη»\\ Μήνας  00, 20XX \\[4\baselineskip]
% \end{flushright}

% Έγινε αποδεκτό από . . . . . . . . . . . . . . . . . . . . . . . . . . . . . . . . . . . . . .

% \begin{flushright}
%     Όνομα Επώνυμο \\ 
%     Ακαδημαϊκός Τίτλος \\
%     Επιβλέπων/ουσα \\[6\baselineskip]
% \end{flushright}

% Έγινε αποδεκτό από . . . . . . . . . . . . . . . . . . . . . . . . . . . . . . . . . . . . . .

% \begin{flushright}
%     Όνομα Επώνυμο \\ 
%     Ακαδημαϊκός Τίτλος \\
%     Μέλος Εξεταστικής Επιτροπής \\[6\baselineskip]
% \end{flushright}

% Έγινε αποδεκτό από . . . . . . . . . . . . . . . . . . . . . . . . . . . . . . . . . . . . . .

% \begin{flushright}
%     Όνομα Επώνυμο  \\ 
%     Ακαδημαϊκός Τίτλος \\
%     Μέλος Εξεταστικής Επιτροπής 
% \end{flushright}


Author . . . . . . . . . . . . . . . . . . . . . . . . . . . . . . . . . . . . . . . . . . . . . . . .\\
\begin{flushright}
    II-MSc “Artificial Intelligence” \\ Month  00, 20XX \\[4\baselineskip]
\end{flushright}

Certified by. . . . . . . . . . . . . . . . . . . . . . . . . . . . . . . . . . . . . . . . . . . . .

\begin{flushright}
    Name Surname \\ 
    Academic Title \\
    Thesis Supervisor \\[6\baselineskip]
\end{flushright}

Certified by. . . . . . . . . . . . . . . . . . . . . . . . . . . . . . . . . . . . . . . . . . . . .

\begin{flushright}
    Name Surname \\ 
    Academic Title \\
    Member of  Examination Committee \\[6\baselineskip]
\end{flushright}

Certified by. . . . . . . . . . . . . . . . . . . . . . . . . . . . . . . . . . . . . . . . . . . . .
\begin{flushright}
    Name Surname \\ 
    Academic Title \\
    Member of  Examination Committee
\end{flushright}

\newpage

%%%%%%%%%%%%%%%%%%%%%%% Uncomment for the Greek Version %%%%%%%%%%%%%%%%%%%%%%%
% \begin{center}
%     \Large{\textbf{Ο Τίτλος Μου}} \\~\\
%     \Large{\textbf{Από}} \\~\\
%     \Large{\textbf{Όνομα Επώνυμο}} \\~\\
    
%     \large{Υποβλήθηκε στο ΔΠΜΣ «Τεχνητή Νοημοσύνη» την \\ ΧΧ Μηνός 20ΧΧ ως \\ υποχρέωση για την λήψη Μεταπτυχιακού Διπλώματος Σπουδών  \\~\\}
% \end{center}

% \renewenvironment{abstract}
%  {\par\noindent\textbf{Περίληψη}\ \ignorespaces}
%  {\par\medskip}

% \begin{abstract}
%     \par Η περίληψη πρέπει να περιλαμβάνει παρουσίαση του προβλήματος που η διπλωματική αντιμετώπισε, τις μεθόδους που ανέπτυξε, τα σημεία συνεισφοράς της, και πως αυτά αναδεικνύονται/αποδεικνύονται από πειραματικά ή/και θεωρητικά αποτελέσματα. \\~\\

%     \begin{flushleft}
%       Επιβλέπων/Επιβλέπουσα:  \\
%       Ακαδημαϊκή Θέση: \\
%     \end{flushleft}
% \end{abstract}

\begin{center}
    \Large{\textbf{My Title}} \\~\\
    \Large{\textbf{By}} \\~\\
    \Large{\textbf{Name Surname}} \\~\\
    
    \large{Submitted to the II-MSc “Artificial Intelligence” on \\ Month XX, 20XX, \\ in partial fulfillment of the \\ requirements for the MSc degree \\~\\}
\end{center}

\renewenvironment{abstract}
 {\par\noindent\textbf{\abstractname}\ \ignorespaces}
 {\par\medskip}

\begin{abstract}
    \\ A small abstract describing the problem addressed, the methods developed, the contributions made and how contributions have are justified through experimental and/or theoretical means. 
    Not more than 20 lines of text. \\~\\
    
    \begin{flushleft}
        Thesis Supervisor:  \\
        Title: \\
    \end{flushleft}
\end{abstract}

\newpage

%%%%%%%%%%%%%%%%%%%%%%% Uncomment for the Greek Version %%%%%%%%%%%%%%%%%%%%%%%

% \section*{Ευχαριστίες}
% \addcontentsline{toc}{section}{Ευχαριστίες}
% \begin{flushleft}
%     Ευχαριστίες προς… 
    
%     Το  υλικό της Διπλωματικής αυτής εργασίας βασίζεται σε εργασία που  πραγματοποιήθηκε και υποστηρίχθηκε από το   «χρηματοδότη»  υπό το συμβόλαιο με αριθμό … 
    
%     Οι  απόψεις που εκφράζονται εδώ , τα ευρήματα και τα συμπεράσματα    είναι  αυτά του / της συγγραφέως και δεν εκφράζουν τις απόψεις του «χρηματοδότη», του Πανεπιστημίου Πειραιώς ή του Ινστ. Πληροφορικής και Τηλεπικοινωνιών  του ΕΚΕΦΕ «Δημόκριτος. 
% \end{flushleft}

\section*{Acknowledgments}
% To be added for EACH un-numbered section.
\addcontentsline{toc}{section}{Acknowledgments}

\begin{flushleft}
    Thank you to ….
    
    This material is based upon work supported by the «Funding Body» Contract No…..
    
    Any opinions, findings, conclusions or recommendations expressed in this material are those of the author(s) and do not necessarily reflect the views of the «funding body» or the view of University of Piraeus and Inst. of Informatics and Telecom. of NCSR “Demokritos”. 
\end{flushleft}

\newpage

\tableofcontents

\subsection*{List of Figures}
\addcontentsline{toc}{subsection}{List of Figures}

\newpage

%%%%%%%%%%%%%%%%%%%%%%% Uncomment for the Greek Version %%%%%%%%%%%%%%%%%%%%%%%

% \section*{Λίστα Πινάκων}
% \addcontentsline{toc}{section}{Λίστα Πινάκων}

\section*{List of Tables}
\addcontentsline{toc}{section}{List of Tables}

\newpage

%%%%%%%%%%%%%%%%%%%%%%% Uncomment for the Greek Version %%%%%%%%%%%%%%%%%%%%%%%
% \section{Εισαγωγή}

% \begin{flushleft}
%     Το παρόν αρχείο αποτελεί ένα αντίγραφο του Πρότυπου συγγραφής διπλωματικής εργασίας που βρίσκεται  \href{http://msc-ai.iit.demokritos.gr/announcements/ekponisi-diplomatikon-ergasion-gia-akadimaiko-etos-2020-2021}{εδώ} σε μορφή LaTeX. \\~\\

%     Το κείμενο που υπάρχει παρακάτω περιέχει κάποιες οδηγίες που αφορούν σε θέματα μορφοποίησης.  
% \end{flushleft}

\section{Introduction}

\begin{flushleft}
    This file is a LaTeX version of the Masters Thesis Template located \href{http://msc-ai.iit.demokritos.gr/announcements/ekponisi-diplomatikon-ergasion-gia-akadimaiko-etos-2020-2021}{here}. \\~\\

    The following section contains some information concerning formatting.  
\end{flushleft}

\newpage

%%%%%%%%%%%%%%%%%%%%%%% Uncomment for the Greek Version %%%%%%%%%%%%%%%%%%%%%%%
% \section{Μορφοποίηση}

% \begin{flushleft}
%     Μεταξύ δύο διαδοχικών τίτλων, ανεξαρτήτως επιπέδου θα πρέπει να υπάρχει κάποιο εισαγωγικό κείμενο 2-3 σειρών (που συνήθως προλογίζει όσα ακολουθούν).  \\~\\
% \end{flushleft}

% \subsection{Γενικές Ρυθμίσεις - Σελιδοποίηση}

% \begin{flushleft}
%     Το τελικό κείμενο θα τυπωθεί σε χαρτί μεγέθους Α4, με εκτύπωση διπλής όψης (εμπρός + πίσω). Τα περιθώρια δεξιά, αριστερά, πάνω και κάτω από το κείμενο είναι 25mm. Επίσης υπάρχει πρόβλεψη για τη βιβλιοδεσία πλάτους 10mm. Το κείμενο θα πρέπει να έχει πλήρη στοίχιση με χρήση συλλαβισμού προκειμένου να αποφεύγονται τα άσχημα μεγάλα κενά στις σειρές (είναι όλα ρυθμισμένα στο παρόν αρχείο). \\~\\

%     Πληροφοριακά, το κυρίως κείμενο είναι σε γραμματοσειρά Georgia με μέγεθος 12 pts και διάστιχο 1½ γραμμής. Για έμφαση του κειμένου θα πρέπει να χρησιμοποιείται μόνο η πλαγιαστή γραφή και ΟΧΙ η έντονη ή η υπογραμμισμένη. \\~\\
    
%     Το παρόν αρχείο είναι σελιδοποιημένο για εκτύπωση διπλής όψης και επιπλέον περιέχει αρίθμηση. Προφανώς, μέχρι την τελική εκτύπωση, μπορείτε να το τυπώνεται και σε εκτύπωση μονής όψης. \\~\\
% \end{flushleft}

% \subsection{Χρήση Style}

% \begin{flushleft}
%     Για ομοιόμορφη μορφοποίηση θα πρέπει να χρησιμοποιήσετε τα styles που περιέχει το παρόν αρχείο. Τα σημαντικότερα από αυτά είναι: \\~\\

%     Το στυλ Normal (Βασικό) για το βασικό κείμενο \\~\\
    
%     Το στυλ Heading 1 (Επικεφαλίδα 1) για την επικεφαλίδα κεφαλαίου, το στυλ Heading 2 (Επικεφαλίδα 2) για την επικεφαλίδα ενότητας, κ.ο.κ. Μπορείτε να χρησιμοποιήσετε μέχρι και το Heading 5. Τα Heading 1 ως 3 έχουν αυτόματη αρίθμηση. Τα 4 και 5 είναι χωρίς αρίθμηση. \\~\\
    
%     Το στυλ Caption (Λεζάντα) που μπαίνει αυτόματα όταν φτιάχνετε λεζάντες. \\~\\
    
%     Τα παραπάνω στυλ είναι ήδη ρυθμισμένα στο παρόν αρχείο, οπότε απλά τα χρησιμοποιείτε.  \\~\\
    
%     Μετά την τελευταία παράγραφο ενότητας, όπως η παρούσα, δεν αφήνουμε γενικά κενές γραμμές καθώς οι παράγραφοι με τους τίτλους είναι ρυθμισμένες έτσι ώστε να δεσμεύουν τον απαιτούμενο χώρο.  \\~\\
% \end{flushleft}

% \subsection{Εικόνες και Πίνακες}

% \subsubsection{Εικόνες}
% \begin{flushleft}
%     Οι εικόνες μπαίνουν in-line και από κάτω έχουν λεζάντα. Για όλες τις εικόνες θα πρέπει να υπάρχει τουλάχιστον μία αναφορά μέσα στο κείμενο. Οι εικόνες θα πρέπει να βρίσκονται κοντά στο κείμενο στο οποίο γίνεται αναφορά σε αυτές και συνήθως μετά από αυτό το κείμενο. Ακολουθεί ένα παράδειγμα τέτοιας αναφορά. \\~\\
% \end{flushleft}

% \subsubsection{Πίνακες}
% \begin{flushleft}
%     Η λεζάντα στον πίνακα μπαίνει στο πάνω μέρος. Μετά τον πίνακα αφήνουμε μία κενή σειρά, όπως στο παράδειγμα.  \\~\\
    
%     \begin{table}
%         \centering
%         \renewcommand\tablename{Πίνακας}
%         \caption{\label{tab:table-name}Δοκιμαστικός πίνακας. }
%         \begin{tabular}{|c|c|c|c|}
%          \hline
%           & Ελλάδα & Αγγλία & Γαλλία\\
%          \hline\hline
%          Πληθυσμός & 10 εκ.  & 55 εκ. & 60 εκ. \\ 
%          \hline
%          Έκταση & 132000 τ.χ. & 800000 τ.χ. & 800000 τ.χ. \\
%          \hline
%         \end{tabular}
%     \end{table}
    
%     Όπως και με τις εικόνες, θα πρέπει να γίνεται αναφορά κάπου μέσα στο κείμενο για τον εκάστοτε πίνακα. Συνίσταται και εδώ η χρήση παραπομπών (cross-reference). \\~\\
% \end{flushleft}

% \subsection{Τίτλοι}

% \begin{flushleft}
%     Οι τίτλοι θα πρέπει να είναι μικροί σε μέγεθος και περιεκτικοί ως προς το περιεχόμενο. Δεν θα πρέπει να ξεπερνούν γενικά τη μία γραμμή. Για τη μορφοποίηση υπάρχουν σχετικά στυλ, όπως αναφέρθηκε. Τα κεφάλαια αρχίζουν σε δεξιά σελίδα, όταν η εκτύπωση είναι διπλής όψης. 
% \end{flushleft}

% \subsection{Βιβλιογραφία }

% \begin{flushleft}
%     Στο τέλος της διπλωματικής θα πρέπει να υπάρχει αριθμημένη βιβλιογραφία. Μέσα στο κείμενο θα πρέπει να βάζετε αναφορά στον αριθμό της πηγής που βρίσκεται στην βιβλιογραφία όπου αυτό είναι απαραίτητο. Η αναφορά αυτή θα γίνεται βάζοντας τον αριθμό της πηγής μέσα σε αγκύλες, π.χ. [1], [2, 3], κτλ... 
% \end{flushleft}

% \subsection{Παράδοση}

% \begin{flushleft}
%     Στο τέλος θα παραδοθούν 1 πρωτότυπο και δύο αντίγραφα της διπλωματικής. Το πρωτότυπο θα έχει επισυναπτόμενο ένα CD που θα περιέχει τον κώδικα, το κείμενο της διπλωματικής και την παρουσίαση της διπλωματικής. 
% \end{flushleft}

\section{Formatting}

\begin{flushleft}
    Μεταξύ δύο διαδοχικών τίτλων, ανεξαρτήτως επιπέδου θα πρέπει να υπάρχει κάποιο εισαγωγικό κείμενο 2-3 σειρών (που συνήθως προλογίζει όσα ακολουθούν).  \\~\\
\end{flushleft}

\subsection{Settings - Paging}

\begin{flushleft}
    Το τελικό κείμενο θα τυπωθεί σε χαρτί μεγέθους Α4, με εκτύπωση διπλής όψης (εμπρός + πίσω). Τα περιθώρια δεξιά, αριστερά, πάνω και κάτω από το κείμενο είναι 25mm. Επίσης υπάρχει πρόβλεψη για τη βιβλιοδεσία πλάτους 10mm. Το κείμενο θα πρέπει να έχει πλήρη στοίχιση με χρήση συλλαβισμού προκειμένου να αποφεύγονται τα άσχημα μεγάλα κενά στις σειρές (είναι όλα ρυθμισμένα στο παρόν αρχείο). \\~\\

    Πληροφοριακά, το κυρίως κείμενο είναι σε γραμματοσειρά Georgia με μέγεθος 12 pts και διάστιχο 1½ γραμμής. Για έμφαση του κειμένου θα πρέπει να χρησιμοποιείται μόνο η πλαγιαστή γραφή και ΟΧΙ η έντονη ή η υπογραμμισμένη. \\~\\
    
    Το παρόν αρχείο είναι σελιδοποιημένο για εκτύπωση διπλής όψης και επιπλέον περιέχει αρίθμηση. Προφανώς, μέχρι την τελική εκτύπωση, μπορείτε να το τυπώνεται και σε εκτύπωση μονής όψης. \\~\\
\end{flushleft}


\subsection{Use of Styles}

\begin{flushleft}
    Για ομοιόμορφη μορφοποίηση θα πρέπει να χρησιμοποιήσετε τα styles που περιέχει το παρόν αρχείο. Τα σημαντικότερα από αυτά είναι: \\~\\

    Το στυλ Normal (Βασικό) για το βασικό κείμενο \\~\\
    
    Το στυλ Heading 1 (Επικεφαλίδα 1) για την επικεφαλίδα κεφαλαίου, το στυλ Heading 2 (Επικεφαλίδα 2) για την επικεφαλίδα ενότητας, κ.ο.κ. Μπορείτε να χρησιμοποιήσετε μέχρι και το Heading 5. Τα Heading 1 ως 3 έχουν αυτόματη αρίθμηση. Τα 4 και 5 είναι χωρίς αρίθμηση. \\~\\
    
    Το στυλ Caption (Λεζάντα) που μπαίνει αυτόματα όταν φτιάχνετε λεζάντες. \\~\\
    
    Τα παραπάνω στυλ είναι ήδη ρυθμισμένα στο παρόν αρχείο, οπότε απλά τα χρησιμοποιείτε.  \\~\\
    
    Μετά την τελευταία παράγραφο ενότητας, όπως η παρούσα, δεν αφήνουμε γενικά κενές γραμμές καθώς οι παράγραφοι με τους τίτλους είναι ρυθμισμένες έτσι ώστε να δεσμεύουν τον απαιτούμενο χώρο.  \\~\\
\end{flushleft}

\subsection{Figures and Tables}

\subsubsection{Figures}

\begin{flushleft}
    Οι εικόνες μπαίνουν in-line και από κάτω έχουν λεζάντα. Για όλες τις εικόνες θα πρέπει να υπάρχει τουλάχιστον μία αναφορά μέσα στο κείμενο. Οι εικόνες θα πρέπει να βρίσκονται κοντά στο κείμενο στο οποίο γίνεται αναφορά σε αυτές και συνήθως μετά από αυτό το κείμενο. Ακολουθεί ένα παράδειγμα τέτοιας αναφορά. \\~\\
\end{flushleft}

\subsubsection{Tables}

\begin{flushleft}
    Η λεζάντα στον πίνακα μπαίνει στο πάνω μέρος. Μετά τον πίνακα αφήνουμε μία κενή σειρά, όπως στο παράδειγμα.  \\~\\
    
    \begin{table}
        \centering
        \renewcommand\tablename{Πίνακας}
        \caption{\label{tab:table-name}Δοκιμαστικός πίνακας. }
        \begin{tabular}{|c|c|c|c|}
         \hline
          & Ελλάδα & Αγγλία & Γαλλία\\
         \hline\hline
         Πληθυσμός & 10 εκ.  & 55 εκ. & 60 εκ. \\ 
         \hline
         Έκταση & 132000 τ.χ. & 800000 τ.χ. & 800000 τ.χ. \\
         \hline
        \end{tabular}
    \end{table}
    
    Όπως και με τις εικόνες, θα πρέπει να γίνεται αναφορά κάπου μέσα στο κείμενο για τον εκάστοτε πίνακα. Συνίσταται και εδώ η χρήση παραπομπών (cross-reference). \\~\\
\end{flushleft}

\subsection{Titles}

\begin{flushleft}
    Οι τίτλοι θα πρέπει να είναι μικροί σε μέγεθος και περιεκτικοί ως προς το περιεχόμενο. Δεν θα πρέπει να ξεπερνούν γενικά τη μία γραμμή. Για τη μορφοποίηση υπάρχουν σχετικά στυλ, όπως αναφέρθηκε. Τα κεφάλαια αρχίζουν σε δεξιά σελίδα, όταν η εκτύπωση είναι διπλής όψης. 
\end{flushleft}

\subsection{References}

\begin{flushleft}
    Στο τέλος της διπλωματικής θα πρέπει να υπάρχει αριθμημένη βιβλιογραφία. Μέσα στο κείμενο θα πρέπει να βάζετε αναφορά στον αριθμό της πηγής που βρίσκεται στην βιβλιογραφία όπου αυτό είναι απαραίτητο. Η αναφορά αυτή θα γίνεται βάζοντας τον αριθμό της πηγής μέσα σε αγκύλες, π.χ. [1], [2, 3], κτλ... 
\end{flushleft}

\subsection{Final Draft}

\begin{flushleft}
    Στο τέλος θα παραδοθούν 1 πρωτότυπο και δύο αντίγραφα της διπλωματικής. Το πρωτότυπο θα έχει επισυναπτόμενο ένα CD που θα περιέχει τον κώδικα, το κείμενο της διπλωματικής και την παρουσίαση της διπλωματικής. 
\end{flushleft}

\newpage

\subsection{Exploring the \texorpdfstring{$\alpha$}{alpha}-Rank Method}

\begin{flushleft}

    Evolutionary dynamics studies how agents' interactions in multi-agent settings evolve over time. While single-agent systems have acquired a strong foundation over the years \cite{10.5555/2831071.2831085}, multi-agent systems are more challenging to analyze.\\~\\

    Current literature indicates a growing interest in studying the evolutionary dynamics of multi-agent systems. Although one might view evolutionary algorithms as mere tools for agents' hyper-parameter tuning \cite{Sinha_2023}\cite{ganapathy2020studygeneticalgorithmshyperparameter}, their contributions extend far beyond that. In the context of games, evolutionary algorithms are widely used to explore game-theoretic concepts. This area of study is also known as \emph{Evolutionary Game Theory}. An example of research in this field is the work of Paul and Deshmukh \cite{paul2022multiagentpathfinding}, who developed an evolutionary algorithm for multi-agent path-finding in stochastic environments. Their approach showed significant improvements in minimizing path length and computational efficiency, outperforming state-of-the-art reinforcement learning algorithms. In another work, David et al. \cite{David_2014} introduced a novel approach for evolving the key components --mainly the evaluation function and search mechanism-- of a chess program from randomly initialized values using genetic algorithms. By learning from databases of grand-master games, their program managed to outperform a world chess champion computer.\\~\\
    
    Building on work done in Evolutionary Game Theory, \emph{$\alpha$-Rank} \cite{omidshafiei2019alpharank} introduces a novel game-theoretic approach to provide insights into the long-term dynamics of agents' interactions. 

    \subsubsection{Markov-Conley Chains}

    \begin{flushleft}

        \emph{$\alpha$-Rank} is an evolutionary methodology designed to evaluate and rank agents' strategies in large-scale multi-agent interactions, using a new dynamic solution concept called \emph{Markov-Conley chains} (MCCs). MCCs extend the concept of Nash equilibrium by providing a more dynamic perspective on stability. While Nash equilibrium focuses on fixed points, where no player has an incentive to deviate, MCCs capture the long-term behavior of agents over time.\\~\\
        
        To better understand the structure of MCCs, we can view the dynamics of agent interactions as flows within a topological space, where each strategy is represented as a point in this space. These flows describe how the system evolves over time, with the state of the system at any given moment depending solely on the current strategy, as defined by the Markov property in Equation~\ref{eq:markov_property}.
        %
        \begin{equation}
            P(s_{t+1} = s_j \mid s_t = s_i) = P(s_{t+1} = s_j \mid s_t = s_i, s_{t-1}, \dots)
            \label{eq:markov_property}
        \end{equation}
        %
        
        where $P(s_{t+1} = s_j \mid s_t = s_i)$ represents the probability of transitioning from strategy $s_i$ to $s_j$ at time $t+1$, independent of past strategies.\\~\\
        
        This concept is further supported by \emph{Conley's Fundamental Theorem of Dynamical Systems}, which divides the state space into recurrent sets, representing stable behaviors, and transient points that eventually lead to recurrent sets. These recurrent sets, including fixed points, periodic orbits, and limit cycles, correspond to different forms of equilibrium. In the context of games, the long-term dynamics of agent interactions are captured in such a space, where each point represents a possible strategy. The graph formed by these strategies consists of SCCs, where each node represents a strategy, and edges represent transitions between strategies. Once players enter these components, they tend to remain, indicating equilibrium \cite{omidshafiei2019alpharank}.\\~\\

        Let $\phi_t: X \rightarrow X$ denote a flow on a topological space $X$. For each time step $t \in \mathbb{R}$, the flow maps a point $x \in X$ to another point in the space, $\phi_t(x)$. This mapping represents the evolution of the agent’s strategy over time, with $x$ being the current strategy and $\phi_t(x)$ the updated one at time $t$. The behavior of agents within the strategy space can be analyzed through the recurrence and stability of points under this flow.
        
        \paragraph{Chain Recurrent Set}

        \begin{flushleft}

            The \emph{chain recurrent set} $\mathcal{R}_\phi$ of the flow $\phi_t$ is the set of all points $x \in X$ that are chain recurrent under the flow $\phi_t$. A point $x$ is chain recurrent if there exists an $(\epsilon, T)$-chain from $x$ to itself, meaning there exists a sequence of points $(x_0, x_1, \dots, x_n)$ connecting back to $x$, with each step being arbitrarily close to the previous one.

            \begin{definition}[Chain recurrent point]
                An $(\epsilon, T)$-chain from $x$ to itself, with respect to the flow $\phi_t$ and distance function $d$, is a sequence $(x_0, x_1, \dots, x_n)$ such that:

                \begin{equation}
                    d(\phi_{t_i}(x_i), x_{i+1}) < \epsilon \quad \text{for each} \quad i = 0, 1, \dots, n-1, \quad t_i \geq T
                    \nonumber
                \end{equation}

                where $\epsilon > 0$ represents the allowed perturbation at each step, and $T > 0$ is the minimum time-step between transitions. If such a chain exists from $x$ back to itself, the point $x$ is chain recurrent.
            \end{definition}

        \end{flushleft}

        \paragraph{Transient Points}

        \begin{flushleft}
            A point $x \in X$ is called \emph{transient} if it is not chain recurrent. This means that there does not exist an $(\epsilon, T)$-chain from $x$ to itself, and trajectories starting from a transient point eventually leave every neighborhood of $x$ without returning. Transient points do not exhibit any form of cyclic behavior; they eventually \emph{escape} from their initial region.
        
            \begin{definition}[Transient point]
                A point $x \in X$ is transient if $x \notin \mathcal{R}_\phi$.
            \end{definition}

        \end{flushleft}

    \end{flushleft}
        
    \subsubsection{Strategy Evolution Process}

    \begin{flushleft}

        Given a K-player game, \emph{$\alpha$-Rank} considers the empirical game with K player slots, called $populations$, where individual agents correspond to strategies, i.e. to styles of playing the underlying game In \emph{$\alpha$-Rank}, populations of agents interact with each other through an evolutionary process following the dynamics of games. The rewards received from these interactions determine how well each strategy performs and, in turn, how often it is adopted by individuals in the populations. Strategies that perform well have a higher probability of being adopted and carried over to the next generation, while those performing poorly are less likely to be adopted. This process of competition and selection between populations leads to their evolution.\\~\\

        To facilitate evolution, \emph{$\alpha$-Rank} uses the concept of mutation. Initially, populations are monomorphic, meaning all individuals within them choose the same strategy. During K-wise interactions, individuals have a small probability of mutating into different strategies or choosing to stick with their current one. The probability that the mutant will take over the population, defined to be the fixation probability function $\rho$, depends on the relative fitness of the mutant and the population being invaded. Fitness is a function that computes the expected reward an individual can receive when adopting a particular strategy, given the strategies of the other individuals. The stronger the fitness, the more likely it is for individuals to mutate, whereas the lower the fitness, the more likely it is for the mutant to go extinct. When the mutation rate is small, we can assume that the fitness for any agent $k$ is:
        %
        \begin{equation}
            f^k(str^{k}, str^{-k}) = P^k(str^{k}, str^{-k}),
            \label{eq:fitness_k}
        \end{equation}
        %

        where $P$ represents the empirical game payoff. Formally, the probability of a mutant strategy $str'$ fixating in some population where individuals play strategy $str$ is given by:
        %
        \begin{equation}
            \rho_{str \to str'} = \frac{1 - e^{-\alpha \cdot \Delta f}}{1 - e^{-\alpha \cdot m \cdot \Delta f}} 
            \label{eq:fixation_prob}
        \end{equation}
        %

        assuming that $\Delta f$ is non-zero. $\Delta f = f^k(str', str^{-k}) - f^k(str, str^{-k})$ represents the difference in fitness between the mutant strategy $str'$ and the resident strategy $str$ in the focal population $k$, while the remaining $K - 1$ populations are fixed in their monomorphic strategies $str^{-k}$. Parameter $m$ is the population size and $\alpha$ is the selection intensity. This adjusts the sensitivity of the system to fitness differences: with higher values of $\alpha$, even small differences in fitness lead to larger changes in $\rho$. The nominator measures the potential of the mutant to ``invade'' the resident population solely based on its fitness advantage. Note that, for example, as $\Delta f$ approaches zero, the probability of the mutant's success decreases. The denominator, on the other hand, normalizes the fixation probability using the population size $m$, making it more challenging for a mutant to dominate in larger populations. When $\Delta f$ is zero, the fixation probability equals to $1/m$, indicating that the mutant strategy has the same probability of taking over as any other strategy in the population. We refer to this probability as the \emph{neutral fixation probability}, denoted by $\rho_m$.

    \end{flushleft}

    \subsubsection{Modeling Dynamics through MCCs}
    
    \begin{flushleft}

        In the context of K-player games, \emph{$\alpha$-Rank} creates a Markov transition matrix over strategy profiles. This is an $|\mathcal{S}tr| \times |\mathcal{S}tr|$ matrix that defines the probability of moving from one strategy profile to another based on how likely each population is to change its strategy.
        %
        \begin{eqnarray}
            C_{str \to str'} = 
            \begin{cases} 
                \eta \cdot \rho_{str \to str'} & \text{if } str \neq str' \\ 
                1 - \sum_{str \neq str'} C_{str \to str'} & \text{otherwise}
            \end{cases} 
            \label{eq:transition_matrix_entry}
        \end{eqnarray}
        %

        Here, $C$ is the strategy-transition matrix where each entry $C_{str \to str'}$ represents the probability of transitioning from strategy $str$ to strategy $str'$. The first part of the formula, calculates the probability of transitioning from one strategy to a different one, scaled to ensure that the sum of probabilities for all possible transitions from that strategy sums up to 1. The second part of the formula, computes the probability of staying with the same strategy, $str$, by excluding transitions to all other strategies.\\~\\

        This evolutionary process of competition and selection among players' strategies leads to a unique stationary probability distribution $\pi$ of dimensionality $|\mathcal{SP}|$, where the mass assigned to a strategy profile indicates how likely it is to resist being ``invaded'' by other strategies as the dynamics evolve. To evaluate and rank strategy profiles —which is the ultimate goal— the method calculates $\pi$ over the game's Markov chain, using the strategy-transition matrix $C$. This distribution indicates how often the system is likely to remain in each profile over time, allowing us to identify the most dominant strategies that are expected to prevail in the long run. Formally, $\pi$ can be computed from the following equation:
        %
        \begin{equation}
            \pi C = \pi \Rightarrow \pi (C - \mathbb{I}) = 0 
            \label{eq:stationary_distribution}
        \end{equation}
        %

        where $\mathbb{I}$ is the identity matrix (see Equation~\ref{eq:c_pi_equation} for the corresponding linear system representation). This means we are looking for a probability vector $\pi$ such that when multiplied by the transition matrix $C$, it remains unchanged. To solve for $\pi$, the augmented matrix from $C - \mathbb{I}$ is constructed and a normalization condition to ensure that probabilities sum to 1 is imposed\footnote{The system $\pi(C - \mathbb{I}) = 0$ by itself does not have a unique solution, as there are infinitely many vectors $\pi$ that satisfy it. To get a unique solution $\pi=(\pi_1,\pi_2,..., \pi_{|\mathcal(SP)|})$, it must hold that $\sum_{i} \pi_i = 1$.}. In this stationary distribution, $\pi=(\pi_1,\pi_2,..., \pi_{|\mathcal(SP)|})$, each $\pi_i$ represents the average time the system spends in strategy profile $i$.
        %
        \begin{equation}
            \begin{bmatrix}
                C_{11} - 1 & C_{12} & \cdots & C_{1n} & | & 0 \\
                C_{21} & C_{22} - 1 & \cdots & C_{2n} & | & 0 \\
                \vdots & \vdots & \ddots & \vdots & | & 0 \\
                C_{n1} & C_{n2} & \cdots & C_{nn} - 1 & | & 0 \\
                1 & 1 & \cdots & 1 & | & 1
            \end{bmatrix}
            \begin{bmatrix}
                \pi_1 \\
                \pi_2 \\
                \vdots \\
                \pi_n \\
                1
            \end{bmatrix}
            =
            \begin{bmatrix}
                0 \\
                0 \\
                \vdots \\
                0 \\
                1
            \end{bmatrix}
        \label{eq:c_pi_equation}
        \end{equation}
        %
    
    \end{flushleft}

\end{flushleft}

\newpage

\subsection{Identifying Strong Joint Policies}

....

\subsubsection{Problem Statement}

\begin{flushleft}
    ....
\end{flushleft}

\subsubsection{Proposed Methodology}

\begin{flushleft}
    ....
\end{flushleft}

\newpage

\subsection{Application on the Graph Coloring Game}

\begin{flushleft}

    In this section, we apply our methodology to the Graph Coloring Game (GCG) defined in Section~\ref{sec:2.1}, to identify strong joint strategies. This analysis allows us to ultimately derive the strategies of the underlying dynamic game, providing an explainable link between the observed styles of play and the actions they represent.\\~\\

    We begin by defining the different styles of play and modeling them as policies using neural network (NN) models. Next, we define the empirical game payoff matrix, which measures how well these policies interact with each other over the long run. Finally, we apply \emph{$\alpha$-Rank} to evaluate and rank the joint policies, gaining insights into the game's dynamics.

    \subsubsection{Defining Styles of Play}
    \label{sec:5.1}

    \begin{flushleft}

        The process of transforming the underlying dynamic graph-coloring game into its empirical form begins by understanding the different ways in which agents respond in the game. This involves two essential steps:
        %
        \begin{enumerate}
            \item Identifying agents' strategies.
            \item Constructing the empirical game payoff matrix.\\~\\
        \end{enumerate}
        %

        Agents' strategies in the game can be thought of as different styles of play, which are typically revealed through their preferences and behaviors in response to the game’s structure. In our experiments, we specify different styles across three main dimensions: color tone (preference for which colors to use), block difficulty (preference for the types of blocks to choose), and coloring approach (preference for the number of colors to use), as shown in Table~\ref{tab:preferences}. These dimensions are combined to define a player's overall style, which can range from complete indifference—where no specific preference is observed in any of the dimensions (denoted by "I")—to highly specific preferences across all dimensions.\\~\\
        %
        \begin{table}[H]
            \centering
            \begin{tabular}{ll}
                \toprule
                \textit{Preference Dimension} & \textit{Value} \\ 
                \midrule
                Color Tone & warm (W) vs. cool (C) \\ 
                Block Coloring Difficulty & small (L) vs. large (A) \\ 
                Coloring Approach & minimalistic (M) vs. extravagant (E) \\ 
                \bottomrule
            \end{tabular}
            \vspace{0.5em}
            \caption{Dimensions specifying agents' strategies}
            \label{tab:preferences}
        \end{table}
        %
       
        In our approach, policies corresponding to specific strategies are represented using convolutional neural networks (CNNs), which are trained to adapt to different styles of play in the graph-coloring game. To guide the training of these policies, we assign specific values to each of the three dimensions of the game's \emph{preference reward}—color tone, block difficulty, and coloring approach. These values, range from -1 to 1, where 1 indicates a strong preference for a particular dimension. For example, a value of 0.7 for warm colors suggests a relatively strong preference for warm tones, while values closer to 0 a tendency towards indifference. In our experimental setting, we define a total of 11 distinct styles of play: I, C, W, E, M, L, A, AE, CA, LE and WL, given ``I'' and combinations of preference values specified in Table~\ref{tab:preferences}. Assuming no inherent bias among the players of the empirical game, we allow populations to sample from the same list of strategies.

    \end{flushleft}

    \subsubsection{Realizing the Empiricl Game Strategies}

    \begin{flushleft}

        All the policy models used to realize the empirical game strategies share a common underlying architecture and training setup. Although hyperparameter tuning is typically recommended, it doesn't make much difference in this case, as these models are relatively easy to optimize when trained in small settings. Regarding the convolutional neural network architecture, it consists of four convolutional layers, each defined with a kernel size of 3, stride of 1, and padding of 1. These parameters are chosen to ensure that the model is capable of extracting spatial features from the input data, which in this case is a grid representation of the game state. The input tensor has dimensions $10 \times 12$, where $|B|=10$ represents the number of blocks in the state and $|CR^*|=12$ represents the number of possible colors a block can have. Each block is encoded using one-hot encoding, meaning that each color is represented as a binary vector of length 12. The output is then flattened and passed through two fully connected layers, which process the data to produce the final output, as shown in Figure~\ref{fig:convDQN}.\\~\\
        %
        \begin{figure}[H]
            \centering
            \includegraphics[width=0.8\linewidth]{images/convDQN.png}
            \caption{Convolution Policy Network Architecture}
            \label{fig:convDQN}
        \end{figure}
        %

        Policy models are trained independently in the underlying game using the deep Q-learning reinforcement learning algorithm specified in Algorithm~\ref{alg:dqn_algorithm}. We set $\gamma$ to 0.7. To optimize the model parameters, we use the smooth L1 loss function with $\beta$=1.0 and the Adam optimizer with a learning rate of 5e-4 and weight decay of 1e-5 to prevent over-fitting. To further enhance the learning process, we incorporate experience replay, with a memory that stores up to 10 million experiences \cite{10.1007/BF00992699}. A target network alongside the main policy network, is being used according to the Double-DQN approach \cite{vanhasselt2015deep}. To update the target network we apply a soft update with a factor $\tau$=5e-3. This gradually brings the target network closer to the policy network, balancing learning speed and stability. With a batch size of 64, we train the models for 10,000 episodes.\\~\\
        %
        \begin{algorithm}
            \caption{Double Deep Q-Learning with Experience Replay}
            \label{alg:dqn_algorithm}
            \begin{algorithmic}[1]
            \State $Q_\theta$, $Q_{\theta'} \gets Q_\theta$, $M$ \Comment{Initialize policy/target nets \& memory}
            \For{episode}
                \State $s \gets s_{0}$
                \For{step}
                    \State $a \gets \text{argmax}_a Q_\theta(s)$ \Comment{Select $\epsilon$-greedy action}
                    \State $(s, a, r, s') \in M$ \Comment{Store experience}
                    \If{$|M| > \text{batch size}$} 
                        \For{each $(s, a, r, s')$ in $M$} \Comment{Sample memory}
                            \State $y \gets r + \gamma \max_{a'} Q_{\theta'}(s')$
                            \State $L \gets \text{Loss}(Q_\theta(s), y)$
                            \State $\theta \gets \theta - \alpha \nabla_\theta L$
                        \EndFor
                    \EndIf
                    \State $Q_{\theta'} \gets \tau Q_\theta + (1 - \tau) Q_{\theta'}$ \Comment{Soft update}
                    \State $s \gets s'$
                \EndFor
            \EndFor
            \end{algorithmic}
        \end{algorithm}
        %
        
        In this setup, the agents are trained without any co-players, meaning that each model learns in isolation. This approach is intentional, as our goal is to develop agents that play optimally on their own, rather than in collaboration with others. The idea is to explore how different styles of independent players interact with each other. We expect that some styles will lead to more conflicts than others, and this behavior is key to our analysis. If we had trained the agents together, for example using a multi-agent reinforcement learning (MARL) approach designed for collaborative settings, the resulting agents would have learned joint policies, which would defeat the very purpose of evaluating how their individual  policies affect collaboration.\\~\\

        Let us consider, for instance, the simulation statistics of two relatively compatible play styles that we expect to perform well together in the game: Player W (a human player with preferences for warm color tones) and Player C (a robot with preferences for cool color tones). In Figure~\ref{fig:cw-solution}, we observe how these players' individual preferences shape their interactions. The statistics in Figure~\ref{fig:cw-stats} reveal the most frequently selected actions in terms of blocks and colors. Player C prefers colors like blue, purple, and green, while Player W tends to choose colors like brown, orange, and red. Given these preferences, conflicts are unlikely to arise from color selection alone. Even in scenarios where they may choose to color neighboring blocks, it is highly impossible that both players will choose the same color.\\~\\
        %
        \begin{figure}[H]
            \centering
            \begin{subfigure}[b]{0.45\textwidth}
                \centering
                \raisebox{0.8cm}[0pt][0pt]{\includegraphics[width=\textwidth]{images/cw-solution.png}}
                \caption{}
                \label{fig:cw-solution}
            \end{subfigure}
            \hspace{0.05\textwidth}
            \begin{subfigure}[b]{0.45\textwidth}
                \centering
                \includegraphics[width=\textwidth]{images/cw-stats.png}
                \caption{}
                \label{fig:cw-stats}
            \end{subfigure}
            \caption{Results from simulating the interactions between players C and W. The left image shows an example solution to the game (a), and the right image presents a statistical analysis of the most frequently selected actions (b).}
            \label{fig:cw-simulation}
        \end{figure}
        %        

        On the other hand, in Figure~\ref{fig:cc-solution}, we observe the interactions between two Player C agents. In this case, we expect more conflicts, as both players share similar color preferences. This increases the likelihood of both players selecting the same color for neighboring blocks. The statistics shown in in Figure~\ref{fig:cc-stats} further support this, as they reveal a high probability of color overlap, primarily due to the dominance of cool colors in both action spaces. However, these conflicts are not a result of insufficient training, but rather stem from the inherent similarity in their preferences.\\~\\
        %
        \begin{figure}[H]
            \centering
            \begin{subfigure}[b]{0.45\textwidth}
                \centering
                \vspace{0.5em}
                \raisebox{0.8cm}[0pt][0pt]{\includegraphics[width=\textwidth]{images/cc-solution.png}}
                \caption{}
                \label{fig:cc-solution}
            \end{subfigure}
            \hspace{0.05\textwidth}
            \begin{subfigure}[b]{0.45\textwidth}
                \centering
                \includegraphics[width=\textwidth]{images/cc-stats.png}
                \caption{}
                \label{fig:cc-stats}
            \end{subfigure}
            \caption{Results from simulating the interactions between players C and C. The left image shows an example solution to the game (a), and the right image presents a statistical analysis of the most frequently selected actions (b).}
            \label{fig:cc-simulation}
        \end{figure}
        % 
        
        Other than that, both agents have been thoroughly trained in isolation, exploring a wide range of states and ultimately converging to stable policies. This allows them to respond effectively and optimally, even when paired with agents with conflicting styles of play. Therefore, even in the case of the most incompatible pairings, the number of mistakes remains minimal, and these mistakes are not due to inadequate exploration, but rather to the inherent conflicts in the agents’ preferences. 

    \end{flushleft}

    \subsubsection{Extracting the Empirical Payoff Matrix}

    \begin{flushleft}

        We generate the empirical payoff matrix by simulating each strategy profile over multiple games. These payoffs represent how well different styles of play perform jointly, according to the game’s rules.\\~\\

        The values in the payoff matrix are computed in terms of the delay and the quality of the solution according to the game's constraints (gain, penalty, and sanction), excluding preferences. This ensures a common ground for distinct strategies, evaluating solutions solely based on the game's rules. For each pair of strategies, we simulate the game over 5,000 repeats and calculate the average payoff for each strategy. These values are then organized into the payoff matrix, which is provided in Table~\ref{tab:gcg_payoff_matrix}. Each entry in the matrix represents the payoffs of strategies in the corresponding profile, with the first value indicating the payoff of the row player and the second value of the column player. The Nash equilibria are highlighted in bold, while nine of the top-ranked strategy profiles in the MCC are shaded in gray.\\~\\
        %
        \begin{table}[h!]
            \centering
            \resizebox{\textwidth}{!}{%
            \begin{tabular}{lccccccccccc}
                \hline
                    & A & AE & C & CA & E & I & L & LE & M & W & WL \\
                \hline
                A  & (3.12, 3.11) & (3.15, 3.16) & (3.17, 3.17) & (3.14, 3.17) & (3.16, 3.17) & (3.16, 3.15) & (3.22, 3.13) & (3.19, 3.16) & (3.15, 3.18) & (3.16, 3.17) & (3.21, 3.18) \\
                
                AE & (3.17, 3.17) & (3.11, 3.11) & (3.18, 3.17) & \cellcolor{gray!16}(3.15, 3.17) & (3.17, 3.16) & (3.19, 3.16) & (3.23, 3.12) & (3.19, 3.16) & \cellcolor{gray!16}(3.15, 3.18) & (3.17, 3.17) & (3.20, 3.16) \\
                
                C  & (3.17, 3.16) & (3.16, 3.17) & (3.10, 3.10) & (3.14, 3.17) & (3.15, 3.15) & (3.18, 3.15) & (3.22, 3.12) & (3.17, 3.14) & (3.14, 3.17) & (3.17, 3.16) & (3.20, 3.17) \\
                
                CA & (3.17, 3.15) & (3.17, 3.15) & (3.17, 3.14) & (3.11, 3.11) & (3.18, 3.15) & (3.18, 3.14) & (3.24, 3.13) & \cellcolor{gray!16}(3.21, 3.16) & \cellcolor{gray!16}(3.16, 3.16) & (3.19, 3.16) & \cellcolor{gray!16}(3.22, 3.15) \\
                
                E  & (3.15, 3.16) & (3.16, 3.16) & (3.15, 3.16) & \cellcolor{gray!16}(3.15, 3.17) & (3.10, 3.10) & (3.18, 3.16) & (3.22, 3.12) & (3.19, 3.14) & (3.15, 3.17) & (3.16, 3.17) & (3.19, 3.17) \\
                
                I  & (3.14, 3.16) & (3.16, 3.18) & (3.16, 3.18) & (3.15, 3.19) & (3.16, 3.17) & (3.12, 3.12) & (3.22, 3.14) & (3.18, 3.16) & (3.14, 3.19) & (3.16, 3.18) & (3.19, 3.18) \\

                L  & (3.14, 3.22) & (3.11, 3.22) & (3.12, 3.22) & (3.13, 3.23) & (3.12, 3.22) & (3.13, 3.22) & (3.12, 3.12) & (3.14, 3.20) & (3.11, 3.21) & (3.14, 3.23) & \textbf{(3.15, 3.21)} \\
                
                LE & (3.15, 3.19) & (3.14, 3.18) & (3.14, 3.18) & (3.15, 3.21) & (3.15, 3.19) & (3.16, 3.17) & (3.20, 3.14) & (3.11, 3.11) & (3.14, 3.22) & (3.15, 3.18) & (3.18, 3.19) \\
                
                M  & (3.17, 3.14) & (3.17, 3.15) & (3.17, 3.15) & \cellcolor{gray!16}(3.16, 3.17) & (3.16, 3.14) & (3.18, 3.14) & (3.23, 3.11) & (3.20, 3.14) & (3.06, 3.08) & (3.18, 3.15) & (3.20, 3.16) \\
                
                W  & (3.17, 3.17) & (3.17, 3.18) & (3.16, 3.18) & \cellcolor{gray!16}(3.16, 3.20) & (3.17, 3.17) & (3.18, 3.16) & (3.21, 3.13) & (3.18, 3.15) & (3.15, 3.18) & (3.08, 3.09) & (3.19, 3.15) \\
            
                WL & (3.17, 3.20) & (3.17, 3.19) & (3.17, 3.19) & (3.17, 3.22) & (3.17, 3.19) & (3.18, 3.19) & \textbf{(3.21, 3.15)} & (3.19, 3.17) & \cellcolor{gray!16}(3.16, 3.20) & (3.16, 3.19) & (3.13, 3.13) \\
                \hline
            \end{tabular}
            }
            \vspace{0.5em}
            \caption{Empirical Payoff Matrix for the Graph Coloring Game}
            \label{tab:gcg_payoff_matrix}
        \end{table}
        %

        From this matrix, we observe that (L, WL) and its symmetric counterpart (WL, L) both with payoffs of (3.15, 3.21) and (3.21, 3.15) respectively, are the only Nash equilibria. It is important to note here that these equilibria prescribe agents' strategies given that they do play the game with rational co-players, but they do not capture the overall dynamics of the game, considering the long-term effects of agents' interactions.

    \end{flushleft}

    \subsubsection{Evaluating and Ranking Joint Policies}

    \begin{flushleft}

        Given the payoff matrix derived from the empirical analysis, we apply the \emph{$\alpha$-Rank} method to evaluate the performance of strategy profiles over time in terms of the MCC solution concept. Specifically, we ran the method 1,000 times, using values of $\alpha$ within the range $[0.1, 10]$ with step=0.01, while assuming populations of size $m=100$. We provide as input the strategies defined in Section~\ref{sec:5.1} and the empirical game payoff matrix. We focus on the rankings of the top 6 strategy profiles, to identify the stronger ones across different values of $\alpha$.\\~\\

        As we observe from the rankings in Table~\ref{tab:ranking_table}, the strategy profile that prevails in the long run is (WL, CA); this is the primary component of the MCC. Although the table was derived using an $\alpha$ value of 2, the rankings remain consistent even when $\alpha$ is set to 10. We choose $\alpha=2$ over $\alpha=10$, to display the rankings of lower-performing strategy profiles, which would otherwise drop to zero. First, it is worth mentioning that the Nash equilibria (L, WL) and (WL, L) don't appear among the top-ranked strategy profiles. This is because MCC components are defined based on how well strategies perform when interacting with other strategies, based on long-term agents interactions. The individual strategies within the Nash equilibrium profile, either WL or L, may not result in favorable interactions with other strategies. As a result, the profile (WL, L) is ranked lower that others.\\~\\
        %
        \begin{table}[H]
            \centering
            \begin{tabular}{lcc}
                \hline
                \textbf{Agent} & \textbf{Rank} & \textbf{Score} \\
                \hline
                (WL, CA) & 1 & 0.42 \\
                (W, CA) & 2 & 0.13 \\
                (M, CA) & 3 & 0.12 \\
                (CA, M) & 4 & 0.08 \\
                (CA, W) & 5 & 0.08 \\
                (CA, LE) & 6 & 0.01 \\
                \hline
            \end{tabular}
            \vspace{0.5em}
            \caption{Rankings for $\alpha=2$}
            \label{tab:ranking_table}
        \end{table}
        %

        To further support our observations regarding the misalignment between the two solution concepts, let's examine why (CA, WL) is part of the MCCs, while (L, WL), the Nash equilibrium, is not. A closer look at the payoff matrix in Table~\ref{tab:gcg_payoff_matrix} reveals that L appears to be the worst-performing strategy for the row player, with an average payoff of 3.13. In this case, being in the Nash equilibrium means the player is stuck with a strategy that gives low rewards, making it the best among other options, rather than a strong choice. If it happens to play this strategy, it would expect its rational opponent to play WL. Strategy CA on the other hand, is the best-performing strategy for the row player, with an average payoff of 3.18. Combined with WL, which is the best performing strategy for the column player, with an average payoff of 3.18, they make profile (CA, WL) becomes the top ranked strategy profile in the ranking Table~\ref{tab:ranking_table}.\\~\\

        Rankings within the MCC are also very intuitive. For example, strategies that prefer different color tones, such as (WL, CA) or (W, CA), tend to result into fewer conflicts since, they naturally avoid selecting the same colors. Similarly, strategies that prefer different blocks based on their difficulty, such as (WL, CA) or (CA, LE), tend to provide solutions with minimal delay, as they naturally avoid coloring the same blocks. Notably, profiles with mixed preferences across these dimensions demonstrate the most promising performance, which explains why (WL, CA), as such a profile, is a key component of the MCC. However, not all profile rankings can be easily explained through the game's rules alone; the expected influence of certain strategies on the quality of the solutions remains ambiguous. For example, profiles with strategies like M and E are more difficult to analyze.\\~\\

        The response graph provides a visualization to interpret the \emph{$\alpha$-Rank} results. This graph illustrates the MCC, using the strategy profiles' masses from the stationary distribution, $\pi$, along with the fixation probability function $\rho$ provided by \emph{$\alpha$-Rank}. Figure~\ref{fig:response_graphs} shows the response graphs for for $\alpha = 0.4$, $1.3$, $1.9$ and $6.4$. We consider it to be part of the descriptive framework $\mathcal{D}$, as it offers insights into how rankings were derived.\\~\\
        %
        \begin{figure}[H]
            \centering
            \begin{subfigure}[b]{0.39\linewidth}
                \includegraphics[width=\linewidth]{images/rg_0.4.png}
                \caption{$alpha=0.4$}
                \label{fig:response_graph_0.4}
            \end{subfigure}
            \hfill
            \begin{subfigure}[b]{0.39\linewidth}
                \includegraphics[width=\linewidth]{images/rg_1.3.png}
                \caption{$alpha=1.3$}
                \label{fig:response_graph_1.3}
            \end{subfigure}
        
            \begin{subfigure}[b]{0.39\linewidth}
                \includegraphics[width=\linewidth]{images/rg_1.9.png}
                \caption{$alpha=1.9$}
                \label{fig:response_graph_1.9}
            \end{subfigure}
            \hfill
            \begin{subfigure}[b]{0.39\linewidth}
                \includegraphics[width=\linewidth]{images/rg_6.4.png}
                \caption{$alpha=6.4$}
                \label{fig:response_graph_6.4}
            \end{subfigure}
        
            \caption{Response graphs of strategy profiles' dynamics.}
            \label{fig:response_graphs}
        \end{figure}
        %

        Each node in the graph represents a unique strategy profile in the MCC, while the edges indicate transitions between them. The values on the edges show the fixation probabilities normalized by the neutral fixation probability, denoted as $\rho_m$. The nodes and edges are color-coded. Darker blue nodes represent more strong joint profiles, while lighter blue nodes represent transient ones. Similarly, bold arrows suggest a strong advantage in shifting between the nodes, whereas faint ones suggest less of an advantage.\\~\\

        The response graph describes the overall dynamics of the strategy profiles in the empirical game. One prominent feature is the primary component of the MCC, specifically the profile (WL, CA). This profile, indicated by a dark blue color, has multiple graph edges leading to it, while none from it, indicating that strategies in this profile are non-transient. This is further supported by the large fixation probabilities along the edges. A particularly prominent example is the cluster (CA, LE)-(CA, M)-(CA, W), which consists of three strongly connected profiles, indicating that once a player adopts one of these profiles, they will likely remain within their cluster. These components reflect stable regions in the game’s strategy dynamics, where transitions between profiles become locked into a cycle.\\~\\

        To further investigate the effect of $\alpha$ on profile dominance, we plotted the stationary distribution $\pi$ across all $\alpha$ values used in the experiments, for the top-performing strategy profiles (see Figure~\ref{fig:alpha_x_pi}). This visualization —also part of $\mathcal{D}$— helps us understand how the stationary distribution changes as the selection intensity increases. The x-axis represents the different $\alpha$ values, ranging from $0.1$ to $3$ in Figure~\ref{fig:alpha_x_pi_3}, and from $0.1$ to $10$ in Figure~\ref{fig:alpha_x_pi_10}, while the y-axis in both figures shows the mass of each strategy profile in the stationary distribution $\pi$. As $\alpha$ increases, the distribution converges, indicating that the selection process stabilizes. The final mass distributions are highlighted in boxed regions. The legend on the right side of the plot displays the top-performing joint strategies, with the stronger ones appearing at the top.\\~\\
        %
        \begin{figure}[H]
            \centering
            \begin{subfigure}[b]{0.49\linewidth}
                \includegraphics[width=\linewidth]{images/alpha_x_pi_3.png}
                \caption{Mass across $\alpha \in [0.1, 3]$}
                \label{fig:alpha_x_pi_3}
            \end{subfigure}
            \hfill
            \begin{subfigure}[b]{0.49\linewidth}
                \includegraphics[width=\linewidth]{images/alpha_x_pi_10.png}
                \caption{Mass across $\alpha \in [0.1, 10]$}
                \label{fig:alpha_x_pi_10}
            \end{subfigure}
            \caption{Effect of ranking intensity $\alpha$ on strategy profile mass in the stationary distribution $\pi$.}
            \label{fig:alpha_x_pi}
        \end{figure}
        %

        We plot two such graphs to observe how the mass of strategy profiles is distributed in the MCCs across different $\alpha$ values. In the stationary distribution resulting from a bigger $\alpha$, the dominant strategy profile (WL, CA) in the MCC achieves a mass of 1, with all other profiles dropping to 0. This is clearly illustrated in the second plot (see Figure~\ref{fig:alpha_x_pi_10}). However, regarding the mass distribution for a smaller range of $\alpha$, depicted n the first plot, the game has not yet converged to the final MCC.

    \end{flushleft}

\end{flushleft}

\newpage

%%%%%%%%%%%%%%%%%%%%%%% Uncomment for the Greek Version %%%%%%%%%%%%%%%%%%%%%%%

% \section{Τίτλος Κεφαλαίου 6}

\section{Section 6}

\newpage

%%%%%%%%%%%%%%%%%%%%%%% Uncomment for the Greek Version %%%%%%%%%%%%%%%%%%%%%%%

% \section{Συμπεράσματα}

\section{Conclusions}
\cite{exampleArticle}

\newpage

%%%%%%%%%%%%%%%%%%%%%%% Uncomment for the Greek Version %%%%%%%%%%%%%%%%%%%%%%%
% \printbibliography[title={Βιβλιογραφία}]

\printbibliography

\newpage

%%%%%%%%%%%%%%%%%%%%%%% Uncomment for the Greek Version %%%%%%%%%%%%%%%%%%%%%%%
% \renewcommand\appendixpagename{Παραρτήματα}

% \begin{appendices}
%     \section*{Παράρτημα A}
% \end{appendices}

\begin{appendices}
    \section*{Appendix A}
    Ενδεικτική Δομή Διπλωματικής. Ισχύει κυρίως για διπλωματικές «Πληροφορικού» περιεχομένου που περιγράφουν κάποια υλοποίηση συστήματος. Τα κεφάλαια 2 και 3 μπορεί να συμπτυχθούν σε ένα, ανάλογα με την περίσταση – ρωτήστε τους επιβλέποντες. Το ίδιο ισχύει και για τα κεφάλαια 5 και 6. \\~\\

    \makebox[\textwidth][c]{
        \begin{tabular}{|c|l|c|}
         \hline
         Κεφάλαια & \multicolumn{1}{|c|}{Περιεχόμενο} & \multicolumn{1}{| p{3cm} |}{\centering Μέγεθος (Σελίδες)} \\
         \hline\hline
         Εξόφυλλο & Βλ. πρότυπο  & 1 \\ 
         \hline
         Αφιέρωση & Προαιρετικό & 1 \\
         \hline
         Πρόλογος & \multicolumn{1}{| p{10cm} |}{Εξήγηση του τίτλου της διπλωματικής. Λίγα λόγια για το σύστημα και γενικά για την επιστημονική περιοχή στην οποία κινείται η εργασία. Που εκπονήθηκε, υπό ποιόν καθηγητή, ευχαριστίες.}  & 1/2-1 \\
         \hline
         Περιεχόμενα & \multicolumn{1}{| p{10cm} |}{Περιεχόμενα (μέχρι και επιπέδου 3)} & 3-4 \\
         \hline
         1 Εισαγωγή & \multicolumn{1}{| p{10cm} |}{Επανάληψη προλόγου με πιο αναλυτικά λόγια Ανάλυση (1 παράγραφος) για το κάθε κεφάλαιο της εργασίας} & 2-5 \\
         \hline
         2 & \multicolumn{1}{| p{10cm} |}{Σύντομη περιγραφή του γενικότερου επιστημονικού/τεχνικού πεδίου που πραγματεύεται η διπλωματική} & 10-15 \\
         \hline
         3 & \multicolumn{1}{| p{10cm} |}{Σύντομη περιγραφή του ειδικότερου θέματος/προβλήματος που πραγματεύεται η διπλωματική} & 10-15 \\
         \hline
         4 & \multicolumn{1}{| p{10cm} |}{Σύντομη περιγραφή του εργαλείου/εργαλείων που χρησιμοποιεί η διπλωματική (αν υπάρχουν)} & 10-15 \\
         \hline
         5 & \multicolumn{1}{| p{10cm} |}{Γενικότερη (αφηρημένη) περιγραφή της λύσης που δόθηκε στο πρόβλημα (π.χ. αρχιτεκτονική του συστήματος)} & 5-10 \\
         \hline
         6 & \multicolumn{1}{| p{10cm} |}{Περιγραφή της υλοποίησης - εικόνες (screenshots), σύντομα τμήματα κώδικα (επεξηγημένα)} & 15-20 \\
         \hline
         7 & \multicolumn{1}{| p{10cm} |}{Επίλογος - Συμπεράσματα - Μελλοντική εργασία Κριτική αναφορά στα πεπραχθέντα της εργασίας Προβλήματα, Δυσκολίες που αντιμετωπίστηκαν Θέματα που δεν λύθηκαν και τίθενται ως μελλοντικός στόχος άλλων διπλωματικών} & 2-5 \\
         \hline
         Βιβλιογραφία & \multicolumn{1}{| p{10cm} |}{Βιβλία - εργασίες - Web links που χρησιμοποιήθηκαν ή κρίνονται απαραίτητα για τον αναγνώστη} & 1-2 \\
         \hline
         Παραρτήματα & \multicolumn{1}{| p{10cm} |}{Κώδικας (ολόκληρος ή σημαντικά μόνο τμήματα αν είναι μεγάλος) User Manual (αν είναι μεγάλο και δεν έχει ενσωματωθεί στο κεφάλαιο της υλοποίησης)} & 30-40 max  \\
         \hline
         \multicolumn{2}{|r|}{\textbf{ΣΥΝΟΛΟ}} & \textbf{90-130} \\
         \hline
        \end{tabular}}
\end{appendices}

\end{document}